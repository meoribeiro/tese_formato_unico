%%%%%%%%%%%%%%%%%%%%%%%%%%%%%%%%%%%%%%%%%%%%%%%%%%%%%%%%%%%%%%%%%
% Capitulo - Resultados
%%%%%%%%%%%%%%%%%%%%%%%%%%%%%%%%%%%%%%%%%%%%%%%%%%%%%%%%%%%%%%%%%

\chapter{metodos}

\vspace{-1cm} 
~
\begin{flushright}
\begin{minipage}[t]{6cm}
{\footnotesize\textit{Passam os s�culos e os homens mas repetem-se os fatos e suas causas.
"}}\\
\rule[0.01mm]{6cm}{0.01mm}\\
{\footnotesize  Gaspar Barlaeus}
\end{minipage}
\end{flushright}
\vspace{0.5cm}


Exemplo de tabela.

\begin{table}
\centering
\caption{Resumos dos resultado obtidos para o periodograma da varia��o temporal da profundidade da linha e uma lista dos per�odos obtidos em ordem de signific�ncia do sinal. $P_{Rot}$ � o per�odo de rota��o em dias e $^{*}$ representa as linhas que sofreram corre��o tel�rica.}
\label{compa}
\begin{tabular}{l|ccc}
\hline \hline \\[-2mm]	
elementos & $\lambda$ (\AA{}) & $P_{Rot}$ dias      &  $\sigma$  \\
\hline \\[-2mm]	



																
$Ti II$ & $5381.0$  & $27.87$  & $96.64$\\


\hline
\hline


\end{tabular}
\end{table}

